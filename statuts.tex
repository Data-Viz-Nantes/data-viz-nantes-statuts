\documentclass[12 pt]{article}

% Hugo, tu peux rechanger, c'était juste pour pouvoir compiler avec pdflatex.
%\usepackage{fontspec}
%\setmainfont{Linux Libertine O}
\usepackage{titlesec}
\usepackage{lastpage}
\usepackage{totcount}
\usepackage[francais]{babel}
\usepackage[utf8]{inputenc}
\usepackage{fullpage}

\pagestyle{empty}
\parskip=8 pt
\raggedright

\titlelabel{Article \thetitle{} — }

\newcommand{\quand}{? février 2024}
\newcommand{\Nom}{\textsc{Data Viz Nantes}}
\newcommand{\Sigle}{DVN}

\regtotcounter{section}

\title{Statuts de l'association \\
  \Nom}
\date{}

\begin{document}

\maketitle

\section{Constitution}
\label{sec:constitution}

Il est fondé entre les adhérents aux présents statuts une association
régie par la loi du 1\ier{} juillet 1901 modifiée par ses textes
d'application.

\section{Dénomination}
\label{sec:denomination}
L'association a pour dénomination~: \og\Nom\fg.

Elle peut être désignée par le sigle~: \og\Sigle\fg.

\section{Objet et moyens}
\label{sec:objet-et-moyens}

L'association \Nom{} a pour objet désintéressé et non lucratif~:

\begin{itemize}
\item la promotion de la visualisation des données, et sa théorie et
  ses techniques, dans la région nantaise
\item l'organisation du Data Viz Nantes Meetup (DVNM) et tout
  autre manifestation qui lui semble utile dans l'achèvement des ses objectifs
\end{itemize}

Les moyens d'action de l'association \Nom{} sont notamment la DVNM et
tout autre moyen qui semble, aux membres de l'association et à son
comité directeur, utile et raisonnable.

\section{Siège social}
\label{sec:siege-social}

Le siège de l'association se situe au ??.

Il pourra être transféré à tout moment par simple décision du conseil
d'administration.

\section{Durée}
\label{sec:duree}

La durée de l'association est illimitée. L'année sociale court du
1\ier{} septembre au 31 août.

\section{Cotisation annuelle}
\label{sec:cotisation-annuelle}

L'existence et le montant de la cotisation annuelle dont chaque membre
doit s'acquitter sont votés pendant l'assemblée générale.

Chaque membre doit régler sa cotisation annuelle au plus tard 3 mois
après le début de l'année sociale (la date limite de paiement est donc
le 30 novembre). Les cotisations seront réglées par virement bancaire
sur le compte bancaire de l'association~;

\section{Composition de l'association}
\label{sec:composition-de-l-association}

L'association se compose de ses membres adhérents.

\section{Admission des membres}
\label{sec:admission-des-membres}

Pour être admis en tant que membre adhérent, il faut~:

\begin{itemize}
\item être parrainé par deux membres de l'association~;
\item avoir été agréé préalablement par le conseil d'administration
  qui, en cas de refus, n'aura pas à en faire connaître les raisons~;
\item formuler et signer une demande écrite~;
\item accepter intégralement les statuts et le règlement intérieur de
  l'association~;
\item s'acquitter du montant de la cotisation annuelle si l'assemblée
  générale en a fixé une, comme prévu par l'article
  \ref{sec:cotisation-annuelle}~;
\item s'engager à participer à la réalisation de l'objet de
  l'association.
\end{itemize}

Toute demande d'agrément d'un nouveau membre (accompagnée de ses
pièces justificatives) devra être adressée au bureau de l'association,
qui la transmettra après vérification de sa conformité au conseil
d'administration. Le conseil d'administration statue sur la demande
d'adhésion et décide d'agréer ou non le postulant. Le refus
d'admission n'a pas à être motivé.

\section{Perte et suspension de la qualité de membre}
\label{sec:perte-et-suspension-de-la-qualite-de-membre}

La qualité de membre de l'association se perd par :

\begin{itemize}
\item démission écrite~;
\item décès~;
\item exclusion prononcée par le conseil d'administration pour les
  motifs suivants~:
  \begin{itemize}
  \item non paiement de la cotisation votée par l'assemblée générale
    trois mois après l'échéance de celle-ci~;
  \item inactivité, définie comme l'absence d'implication dans
    l'association pendant une durée d'un an~;
  \item tout autre motif grave laissé à son appréciation. Est
    considéré comme motif grave toute initiative directe ou indirecte
    d'un membre visant à diffamer l'association ou certains de ses
    membres ou à porter atteinte aux objectifs poursuivis par
    l'association.
  \end{itemize}
\end{itemize}

S'il le juge opportun, le conseil d'administration peut décider, pour
les mêmes motifs que ceux indiqués ci-dessus, la suspension temporaire
d'un membre plutôt que son exclusion.  Cette décision implique la
perte de la qualité de membre et du droit de participer à la vie
sociale, pendant toute la durée de la suspension, telle que déterminée
par le conseil d'administration dans sa décision. Si le membre
suspendu est investi de fonctions électives, la suspension entraîne
également la cessation de son mandat.

\section{Conseil d'Administration}
\label{sec:administration}

L’association est dirigée par un conseil de 10 membres au maximum,
élus pour 2 années, renouvelés en moitié chaque année par l’assemblée
générale. Les membres sont rééligibles.

En cas de vacance, le conseil pourvoit provisoirement au remplacement
de ses membres. Il est procédé à leur remplacement définitif à la
plus prochaine assemblée générale. Les pouvoirs des membres ainsi élus
prennent fin à l’expiration du mandat des membres remplacés.

Le conseil d’administration se réunit au moins une fois tous les six
mois, sur convocation d’au moins un membre du bureau, ou à la demande
du quart de ses membres.

Les décisions sont prises à la majorité des voix; en cas de partage,
la voix de la majorité des membres du bureau est prépondérante; en cas
de partage du bureau, la décision n’est pas prise mais est soumise à
discussion ultérieure.

Le Conseil d’administration peut déléguer des pouvoirs pour une durée
déterminée, à un ou plusieurs de ses membres.

Tout membre du conseil qui, sans excuse, n’aura pas assisté à trois
réunions consécutives sera considéré comme démissionnaire.

\section{Bureau}
\label{sec:reunion-du-bureau}

Se le conseil d'administration est composé de plus de 4~personnes, il
élit parmi ses membres, un bureau qui agit de manière collégiale.

Le bureau se réunit aussi souvent que l'exige l'intérêt de
l'association.

\section{Réunion du conseil d'administration}
\label{sec:reunion-du-conseil-d-administration}

Le conseil d'administration se réunit sur convocation d'au moins deux
de ses membres ou aussi souvent que l'exige l'intérêt de
l'association. La présence de la moitié des membres du conseil
d'administration est nécessaire pour la validité des délibérations. Si
le quorum n'est pas atteint lors de la réunion du conseil
d'administration, ce dernier sera convoqué à nouveau à une semaine
d'intervalle, et il pourra valablement délibérer, quels que soient le
nombre de membres présents.

Les décisions sont prises à la majorité absolue des membres présents
ou représentés à moins que le nombre de voix qui s'abstiennent ne dépasse
la moyenne du nombre de vote pour toute option sauf celle recevant la
majorité relative.  Dans ce dernier cas, le vote est considéré comme
repoussant les résolutions mises au vote.

En cas de partage, la décision n'est pas prise.

Tout membre du conseil d'administration, qui, sans excuse, n'aura pas
assisté à trois réunions consécutives pourra être considéré comme
démissionnaire.

Il est tenu procès-verbal des séances.

\section{Pouvoir}
\label{sec:pouvoir}

Le conseil d'administration est investi des pouvoirs les plus étendus
pour faire ou autoriser tous les actes ou opérations dans la limite de
son objet et qui ne sont pas du ressort de l'assemblée générale.

Il autorise le président à agir en justice.

Il surveille la gestion des membres du bureau et a le droit de se
faire rendre compte de leurs actes.

Il arrête le budget et les comptes annuels de l'association.

Cette énumération n'est pas limitative.

Il peut faire toute délégation de pouvoirs pour une question
déterminée et un temps limité.

\section{Assemblée générale}
\label{sec:assemblee-generale}


L’assemblée générale ordinaire comprend tous les membres de
l’association. Elle se réunit chaque année, à une date fixée par le
conseil d’administration.

Un membre du bureau, assisté des
membres du conseil, préside l’assemblée et expose la situation morale
ou l’activité de l’association.

Un membre du bureau rend compte de
sa gestion et soumet les comptes annuels (bilan, compte de résultat et
annexe) à l’approbation de l’assemblée.

L’assemblée générale fixe le montant des cotisations annuelles.

Les décisions sont prises à la majorité relative des voix des membres
ayant le droit de vote et présents ou représentés. Un membre ne peut
représenter plus de quatre autres membres.

Il est procédé, après épuisement de l’ordre du jour, au renouvellement
des membres sortants du conseil.

Toutes les délibérations sont prises à main levée sauf sur demande
d’au moins deux membres présents.

Les décisions des assemblées générales s’imposent à tous les membres,
y compris absents ou représentés.

Si le quorum n'est pas atteint lors de la réunion de l'assemblée, sur
première convocation, l'assemblée sera convoquée à nouveau à quinze
jours d'intervalle et, lors de cette nouvelle réunion, elle pourra
valablement délibérer quel que soit le nombre de membres présents ou
représentés.


\section{Assemblée générale extraordinaire}
\label{sec:assemblee-generale-extraordinaire}

Si besoin est, ou sur la demande d’un quart des adhérents, tout membre
du bureau peut convoquer une assemblée générale extraordinaire,
suivant les modalités prévues aux présents statuts.

L'assemblée générale extraordinaire est seule compétente pour modifier
les statuts, prononcer la dissolution de l'association et statuer sur
la dévolution de ses biens, décider de sa fusion avec d'autres
associations ou sa transformation.

Les modalités de décision sont les mêmes que celles prévues par
l'article \ref{sec:assemblee-generale} pour les assemblées générales.

Une feuille de présence sera émargée et certifiée par les membres du bureau.

Si le quorum n'est pas atteint lors de la réunion de l'assemblée, sur
première convocation, l'assemblée sera convoquée à nouveau à quinze
jours d'intervalle et, lors de cette nouvelle réunion, elle pourra
valablement délibérer quel que soit le nombre de membres présents ou
représentés.

\section{Procès-verbaux des assemblées générales}
\label{sec:proces-verbaux-des-assemblees-generales}

Les délibérations des assemblées sont constatées sur des
procès-verbaux contenant le résumé des débats, le texte des
délibérations et le résultat des votes.

Ils sont commité sur un repo github de l'association prévu à cet
effet.


\section{Dissolution}
\label{sec:dissolution}

La dissolution de l'association ne peut être prononcée que par
l'assemblée générale extraordinaire, convoquée spécialement à cet
effet et statuant aux conditions de quorum et de majorité prévues à
l'article \ref{sec:assemblee-generale-extraordinaire}.

L'assemblée générale extraordinaire désigne un ou plusieurs
liquidateurs chargés des opérations de liquidation.

Lors de la clôture de la liquidation, l'assemblée générale
extraordinaire se prononce sur la dévolution de l'actif net au profit
de toutes associations déclarées de son choix, ayant un objet
similaire.

\section{Ressources}
\label{sec:ressources}

Les ressources de l'association sont toutes celles qui ne sont pas
interdites par les lois et règlements en vigueur.

\section{Règlement intérieur}
\label{sec:reglement-interieur}

Le conseil d'administration pourra, s'il le juge nécessaire, arrêter
le texte d'un règlement intérieur, qui détermine les détails
d'exécution des présents statuts.

Ce règlement sera soumis à l'approbation de l'assemblée générale,
ainsi que ses modifications éventuelles.

\section{Formalités}
\label{sec:formalites}

Les membres du bureau, au nom du conseil d'administration, sont
chargés de remplir toutes formalités de déclarations et publications
prescrites par le législateur.

\vfill{}

\begin{flushright}
  À Nantes, le \quand{},
\end{flushright}

\vspace{10mm}

\hfile{}Line Ton That\hfill{}Ange Ouya \hspace{4cm} Jeffrey Abrahamson\hfill{}

\vspace{15mm}

Ce document relatif aux statuts de l'association \Nom{} comporte
\pageref{LastPage} pages, ainsi que \total{section} articles.

\vfill{}

% Cf. également https://www.alchemistowl.org/pocorgtfo/pocorgtfo14.pdf
% Voir le MD5 en bas de la première page, puis la discussion qui
% commence sur la page 46 ff (14:09 ff).

\end{document}

%%% Local Variables:
%%% mode: latex
%%% TeX-master: t
%%% End:
